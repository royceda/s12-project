\documentclass[11pt]{report}

\usepackage[french]{babel}
\usepackage[latin1]{inputenc}
\usepackage{amsmath}
\usepackage{graphicx}

\usepackage{tikz}
\usetikzlibrary{arrows}


\title{\LARGE Projet: E-Commerce }
\author{Boudjeltia Reda, Adotevi Lionel, Sabir Reda }

\begin{figure}
  \centering
  \includegraphics[scale=0.35]{enseirb.jpg}
  \label{fig:enseirb}
\end{figure}



%Corps du document :

\begin{document}
\maketitle

\tableofcontents


\chapter*{Introduction}

Le projet se situe dans le cadre d'un apprentissage du cour de SGBD, � l'issue de la formation du semestre 7, des sujets de projets on �t� propos�. Parmis ces sujets on y trouve le covoiturage, le sport et le e-commerce. Notre choix s'est orient� vers le e-commerce, puisque c'est un domaine tr�s dynamique de nos jours. Durant presque deux mois nous avons concu une base de donn�es d'un site de E-commerce de vetement g�rant des produits, des promos, des statistiques etc...

\chapter{Contexte du Projet et besoins}


\chapter{Partie Base de donn\'ee}

\section{Mod\'elisation conceptuelle}
\section{Mod\'elisation relationnelle}
\section{R\'ealisation SQL}
\subsectio{Les tables et contraintes}
\subsection{Les vues}
\section{Les triggers}


\chapter{Partie conception Web}

\section{}
\subsection{}



\chapter{Remarques et am\'eliorations possibles}




    



\chapter{Conclusion}
\end{document}

