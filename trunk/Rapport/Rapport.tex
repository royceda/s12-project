\documentclass[11pt]{report}

\usepackage[french]{babel}
\usepackage[latin1]{inputenc}
\usepackage{amsmath}
\usepackage{graphicx}

\usepackage{tikz}
\usetikzlibrary{arrows}


\title{\LARGE Projet: E-Commerce }
\author{Boudjeltia Reda, Adotevi Lionel, Sabir Reda }

\begin{figure}
  \centering
  \includegraphics[scale=0.35]{enseirb.jpg}
  \includegraphics[scale=0.35]{envie2E.jpg}
  \label{fig:enseirb}
\end{figure}



%Corps du document :

\begin{document}
    \maketitle
   
    \tableofcontents


\chapter*{Introduction}
    
      
\chapter{Pr�sentation de l'entreprise}


\chapter{Pr\'esentation du probl\`eme}
   
    \section{Objectifs}
    \section{Situation initiale et contraintes}
   
   
\chapter{Traitement du projet} 
  
   \section{Mod\'elisation}

       \subsection{La structure events}


\begin{figure}
\centering
  \includegraphics[scale=0.8]{event.png}
  \label{fig:event}
\end{figure}

\newpage


       \subsection{La structure camion}
%image de la fenetre d'entr�e des donn�es



       \subsection{Les calendriers de camions}


\newpage
    \section{Sauvegarde des donn\'ees}


    \section{L'interface Graphique}

    \subsection{Pr�sentation de l'interface}    

\begin{figure}
\left
  \includegraphics[scale=0.25]{interface.png}
  \label{fig:interface}
\end{figure}
\newpage


  

    
    



    \chapter{Conclusion}
\end{document}
  
